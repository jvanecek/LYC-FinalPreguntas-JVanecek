\usepackage{amsmath}
\usepackage{amsthm}
\usepackage{amssymb}
\usepackage{amsfonts}
\usepackage{amssymb}
\usepackage[spanish, activeacute]{babel}
\usepackage[usenames,dvipsnames]{color}
\usepackage[width=15.5cm, left=3cm, top=2.5cm, height= 24.5cm]{geometry}
\usepackage{graphicx}
\usepackage[utf8]{inputenc}
\usepackage{listings}
\usepackage[all]{xy}
\usepackage{multicol}
\usepackage{subfig}
\usepackage{algorithm}
\usepackage{algorithmic}
\usepackage{cancel}
\usepackage{float}
\usepackage{xcolor}
\usepackage{color,hyperref}
\usepackage{epsfig}
\usepackage{paralist}

%% para letras caligraficas
\usepackage{mathrsfs}

%%%%%%%%%%%%%% ALGUNAS MACROS %%%%%%%%%%%%%%
% For \url{SOME_URL}, links SOME_URL to the url SOME_URL
\providecommand*\url[1]{\href{#1}{#1}}

% Same as above, but pretty-prints SOME_URL in teletype fixed-width font
\renewcommand*\url[1]{\href{#1}{\texttt{#1}}}

% Comando para poner el simbolo de Reales
\newcommand{\real}{\hbox{\bf R}}

\providecommand*\code[1]{\texttt{#1}}

%uso: \ponerGrafico{file}{caption}{scale}{label}
\newcommand{\ponerGrafico}[4]
{\begin{figure}[H]
	\centering
	\subfloat{\includegraphics[scale=#3]{#1}}
	\caption{#2} \label{fig:#4}
\end{figure}
}

\newcommand{\tomadoEl}[1]{\textit{Tomado el: #1}}

\renewcommand{\algorithmiccomment}[1]{\hfill #1}

\setlength{\parindent}{0cm} % anulo la indentacion en todo el texto
\newtheoremstyle{miestilo}
	{1cm} % Space above
	{} % Space below
	{} % Body font
	{} % Indent amount
	{\bfseries} % Theorem head font
	{\newline} % Punctuation after theorem head
	{1cm} % Space after theorem head
	{} % Theorem head spec (can be left empty, meaning `normal')

\theoremstyle{miestilo}
\newtheorem{definicion}{Definici\'on}
\newtheorem{propiedad}{Propiedad}
\newtheorem{teorema}{Teorema}
\newtheorem{lema}{Lema}
\newtheorem{corolario}{Corolario}
\newtheorem{algoritmo}{Algoritmo}
\newtheorem{proposicion}{Proposici\'on}
