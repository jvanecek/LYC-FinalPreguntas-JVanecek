
\section{Computabilidad}

\begin{questions}
\question Probar que la clase de funciones computables es una clase PRC.
\begin{solution}
  
  Idea: una clase de funci'on es PRC si est'an las iniciales y el resto se forma por composici'on o recursi'on. Lo que hay que demostrar es que las computables cumplen esto. 

  {\it Demostraci\'on: } Las funciones iniciales son computables, porque

  \begin{itemize}
  \item $n(x) = 0$ se computa con el programa vac'io. 
  \item $s(x) = x+1$ se computa con el programa $\{ Y \leftarrow X+1 \}$. 
  \item $u_i(x_1, \dots, x_n) = x_i$ se computa con el programa $\{ Y \leftarrow X_i \}$. 
  \end{itemize}

  Falta ver que la clase de funciones computables est'a cerrada por \begin{inparaenum}[(a)] \item composici'on; y \item recursi'on primitiva. \end{inparaenum}

  \begin{enumerate}[(a)]
  \item Si $h$ se obtiene a partir de las funciones (parciales) computables $f, g_1, \dots, g_k$ por composici'on, entonces $h$ es computable. Y esto es cierto porque el siguiente programa computa $h$: 
  
  \vspace{0.5cm}
  \begin{tabular}{l}
      $Z_1 \leftarrow g_1(X_1, \dots, X_n)$ \\
      $\vdots$ \\
      $Z_k \leftarrow g_k(X_1, \dots, X_n)$ \\
      $Y \leftarrow f(Z_1, \dots, Z_k)$
  \end{tabular}
  \vspace{0.5cm}
  
  Y si $f, g_1, \dots, g_k$ son computables, entonces $h$ es computable.
  
  \item Si $h$ se obtiene a partir de $g$ por recursi'on primitiva y $g$ es computable entonces $h$ es computable. 
  
  En efecto, el siguiente programa computa $h$: 
  
  \vspace{0.5cm}
  \begin{tabular}{rl}
	  & $Y \leftarrow k$ \\ 
    $[A]$ & IF $X=0$ GOTO $E$ \\
	  & $Y \leftarrow g(Z,Y)$ \\
	  & $Z \leftarrow Z+1$ \\
	  & $X \leftarrow X-1$ \\
	  & GOTO $A$
  \end{tabular}
  \vspace{0.5cm}
  
  Y por lo tanto, si $g$ es computable, $h$ tambi'en. 
  \end{enumerate}
\end{solution}

\question Probar que una funci'on es primitiva recursiva sii pertenece a toda clase PRC.
\begin{solution}
  
  {\it Demostraci\'on: } Una funci\'on es p.r. si se puede obtener a partir de las funciones iniciales por un n\'umero finito de aplicaciones de composici\'on y recursi\'on primitiva. Es la clas de funciones PRC m\'as chica. 
  
  \begin{itemize}
   \item[$\Leftarrow$)] Como la clase de funciones p.r. es una clase PRC, luego si $f$ pertenece a toda clase PRC, en particular $f$ es p.r.
   
   \item[$\Rightarrow$)] Supongamos que existe $f$ p.r. y una clase $C$ PRC. 

    Por definici'on de p.r., $f$ se puede obtener a partir de funciones inciales y por un n'umero finito de aplicaciones de composici'on y recursi'on primitiva. Es decir, que existe una lista de funciones $f_1, f_2, \dots, f_n$ tal que: 

    \begin{itemize}
    \item $f_i$ o bien es una funci'on inicial (y por lo tanto est'a en $C$) o bien se obtiene por composici'on o recursi'on primitiva a partir de funciones $f_j$, con $j < i$ (y por lo tanto est'a en $C$). 
    \item $f_n = f$
    \end{itemize}

    Por inducci'on, todas las funciones de la lista est'an en $C$. En particular $f_n = f \in C$. 
  \end{itemize}

\end{solution}

\question Probar que la clase de funciones p.r. es la clase PRC m\'as chica. 

\begin{solution}
 Se deduce de demostrar que una funci\'on es p.r. sii pertenece a toda clase PRC. 
\end{solution}

\question Sean $A$ y $B$ dos conjuntos c.e. Demostrar que $A \cup B$ y $A \cap B$ son c.e. ?`Es $A\backslash B$ c.e.?
\begin{solution}
  
  {\it Definici\'on:} Un conjunto $A$ es c.e. cuando existe una funci'on parcial computable $g: \mathbb{N} \rightarrow \mathbb{N}$ tal que: 

  \begin{equation*}
  A = \{ x : g(x) \downarrow \} = \text{dom}(g)
  \end{equation*}

  (En otras palabras, podemos decidir algor'itmicamente si un elemento s'i pertenece a $A$, aunque para elementos que no pertenecen a $A$ el algoritmo se indefine)

  {\it Demostraci\'on: $A \cap B$}

  Sean $p$ y $q$ los n'umeros de los programas tales que

  \begin{center}
  \begin{tabular}{ l r }
  \(\displaystyle A = \{ x : \Phi_p(x) \} \) & \(\displaystyle B = \{ x : \Phi_q(x) \} \) \\
  \end{tabular}
  \end{center}

  Si podemos encontrar un programa que se defina con las mismas entradas que definen a $A$ y $B$, entonces estamos. 

  Sea $R$ el siguiente programa: 

  \vspace{0.5cm}
  \begin{tabular}{l}
  $Y \leftarrow \Phi_p(x)$ \\ 
  $Y \leftarrow \Phi_q(x)$
  \end{tabular}
  \vspace{0.5cm}

  Notemos que $\Psi_R(x)\downarrow$ si y solo si $\Phi_p(x)\downarrow$ y $\Phi_q(x)\downarrow$. 

  En efecto, $A\cap B$ es c.e. porque

  \begin{equation*}
  A\cap B = \{ x : \Psi_R(x)\downarrow \}
  \end{equation*}

  {\it Demostraci\'on: $A\cup B$}

  Tengo que encontrar un programa que me permita decidir si un elemento pertenece a $A$ o a $B$. 

  Para ello vamos a usar la funci'on $STP^{(n)}(x_1, \dots, x_n, e, t)$ que es $TRUE$ si y solo si el programa con n'umero $e$ termina en $t$ o menos pasos con las entradas $x_1, \dots, x_n$. 

  Definimos $R$ como: 

  \vspace{0.5cm}
  \begin{tabular}{rl}
  $[C]$ & IF $STP^{(1)}(X, p, T)$ GOTO $E$ \\
      & IF $STP^{(1)}(X, q, T)$ GOTO $E$ \\
      & $T \leftarrow T + 1$ \\
      & GOTO $C$
  \end{tabular}
  \vspace{0.5cm}

  Notemos que este programa chequea \emph{al mismo tiempo} si un elemento pertenece al dominio de $\Phi_p$ o de $\Phi_q$, ejecutando paso a paso los programas $p$ y $q$. 

  En efecto $\Psi_R(x)\downarrow$ si y solo si $\Phi_p(x)\downarrow$ o $\Phi_q(x)\downarrow$: 

  \begin{equation*}
  A\cup B = \{ x : \Psi_R(x)\downarrow \}
  \end{equation*}

  {\it Demostraci\'on: $A\backslash B$}

  Supongamos que $A\backslash B$ fuera c.e. Sea $A$ el conjunto de todos los programas, y $B$ s\'olo de aquellos que paran cuando se les da como entrada su n\'umero de programa. $A\backslash B$ es entonces el conjunto de programas que no se detienen cuando se les da su n\'umero de programa como entrada. 
  
  Por definici\'on, un conjunto es c.e. si podemos decidir algor\'itmicamente si un elemento pertenece o no. Es decir que HALT($X,X$) ser\'ia computable. Absurdo! Y esto vino de suponer que $A\backslash B$ era c.e.
  
\end{solution}

\question Seas $A$ y $B$ conjuntos de una clase PRC $C$. Entonces $A\cup B$, $A\cap B$, y $\bar{A}$ est'an en $C$.

\question Definir conjunto de índices. Enunciar y demostrar el Teorema de Rice.
\begin{solution}
  
  {\it Definici'on:} un conjunto de naturales $A$ es un conjunto de 'indices si existe una clase de funciones $\mathbb{N} \rightarrow \mathbb{N}$ parciales computables $C$ tal que $A = \{ x : \Phi_x \in C \}$. 

  {\it Teorema de Rice:} Si $A$ es un conjunto de 'indices tal que $\emptyset \neq A \neq \mathbb{N}$, $A$ no es computable. 

  {\it Demostraci'on:} Supongamos la clase de funciones $C$ tal que $A = \{ x : \Phi_x \in C \}$ computable. Sea $f \in C$ y $g \notin C$ funciones parciales computables. Sea $h : \mathbb{N}^2 \rightarrow \mathbb{N}$ la siguiente funci'on parcial computable: 

  $$
  h(t,x) = \left\{
  \begin{array}{cl}
  g(x) & \mbox{si } t \in A \\
  f(x) & \mbox{si no}
  \end{array}\right.
  $$

  Por el teorema de la Recursi'on, existe $e$ tal que $\Phi_e(x) = h(e,x)$. Luego, 

  \begin{itemize}
  \item Si $e\notin A \Rightarrow h(e,x) = f(x)\Rightarrow \Phi_e=h=f\Rightarrow$ como $f\in C, \Phi_e \in C \Rightarrow e \in A$. Absurdo! 
  \item Si $e\in A \Rightarrow h(e,x) = g(x)\Rightarrow \Phi_e=h=g\Rightarrow$ como $g\notin C, \Phi_e \notin C \Rightarrow e \notin A$. Absurdo! 
  \end{itemize}

  \end{solution}

  \question Enunciar y demostrar el Teorema de la Recursi'on.
  \begin{solution}
  
  {\it Teorema de la Recursi\'on:} Si $g : \mathbb{N}^{n+1} \rightarrow \mathbb{N}$ es parcial computable, existe un $e$ tal que

  \begin{equation*}
  \Phi_e(x_1, \dots, x_n) = g(e, x_1, \dots, x_n)
  \end{equation*}

  {\it Demostraci'on:} Sea $S_n^1$ la funci'on del teorema del Par'ametro, tal que: 

  \begin{equation*}
  \Phi_y^{(n+1)}(x_1, \dots, x_n, u) = \Phi_{S_n^1(u,y)}^{(n)}(x_1, \dots, x_n)
  \end{equation*}

  La funci'on $(x_1, \dots, x_n, v) \mapsto g(S_n^1(v,v), x_1, \dots, x_n)$ es parcial computable, de modo que existe $d$ tal que: 

  \begin{eqnarray*}
  g(S_n^1(v,v), x_1, \dots, x_n) &=& \Phi_d^{(n+1)}(x_1, \dots, x_n, v) \\
				  &=& \Phi_{S_n^1(v,d)}^{(n)}(x_1, \dots, x_n)
  \end{eqnarray*}

  Como $d$ est'a fijo, y $v$ es variable, elegimos $v = d$ y $e = S_n^1(d,d)$. Luego

  \begin{eqnarray*}
  g(S_n^1(v,v), x_1, \dots, x_n) &=& \Phi_{S_n^1(v,v)}^{(n)}(x_1, \dots, x_n) \\
	    g(e, x_1, \dots, x_n) &=& \Phi_e(x_1, \dots, x_n)
  \end{eqnarray*}
\end{solution}

\question Enunciar y demostrar el Teorema de Punto Fijo.

\begin{solution}

 {\it Teorema del Punto Fijo: } Si $f : \mathbb{N}\rightarrow\mathbb{N}$ es computable, existe $e$ tal que $\Phi_{f(e)}=\Phi_e$.
 
 {\it Demostraci\'on: } Considerar la funci\'on $g : \mathbb{N}^2\rightarrow\mathbb{N}$, $g(z,x) = \Phi_{f(z)}(x)$.
 
 Aplicando el Teorema de la Recursi\'on, existe un $e$ tal que para todo $x$, 
 
 \begin{equation*}
  \Phi_e(x)=g(e,x)=\Phi_{f(e)}(x)
 \end{equation*}

\end{solution}

\question Enunciar Halt y demostrar que no es computable (con cualquiera de las demostraciones vistas).

\begin{solution}

El problema Halt consiste en tratar de determinar si un programa $P$ con n\'umero de programa $y$, y con entrada $x$ termina. 

Y ello se define con la funci\'on: 
$$
HALT(x,y) = \left\{
\begin{array}{cl}
1 & \mbox{si } \Phi_y^{(1)}(x)\downarrow \\
0 & \mbox{si } \Phi_y^{(1)}(x)\uparrow
\end{array}\right.
$$

Supongamos que HALT es computable. Construimos el siguiente programa $Q$: 

\begin{tabular}{rl}
  $[A]$ & IF HALT($X,X$) = 1 GOTO $A$ 
\end{tabular}

Supongamos que $\#Q=e$. Entonces

$$
\Phi_e(x) = \left\{
\begin{array}{cl}
\uparrow & \mbox{si } HALT(x,x)=1 \\
0 & \mbox{si no} 
\end{array}\right.
$$

Entonces

$$
\text{HALT}(x,e)=1 \quad\Leftrightarrow\quad \Phi_e(x)\downarrow \quad\Leftrightarrow\quad \text{HALT}(x,x)\neq 1
$$

Si bien $e$ est\'a fijo, $x$ es variable. Llegamos a un absurdo con $x=e$. 
\end{solution}

\question Exhibir una función computable que no sea p.r. y demostrarlo.

\begin{solution}

Sea la funci\'on $\tilde{\Phi}_e^{(n)}(x_1,\dots,x_n)$ computable que simula a la $e$-\'esima funci\'on p.r. con entrada $x_1, \dots, x_n$.

Sea $g : \mathbb{N} \rightarrow \mathbb{N}$, definida como $g(x) = \tilde{\Phi}_x^{(1)}(x)$. Demostremos que $g$ es computable pero no p.r.

{\it Demostraci\'on: } Claramente $g$ es computable porque $\tilde{\Phi}$ lo es. 

Supongamos que es p.r. $\Rightarrow f(x) = g(x)+1$ es p.r. por composici\'on $\Rightarrow \exists e$ tal que $\tilde{\Phi}_e = f$ por ser $f$ p.r. $\Rightarrow\tilde{\Phi}_e(x) = \tilde{\Phi}_x(x)+1$.

$e$ est\'a fijo pero $x$ es variable. As\'i que instanciando $x=e$, llegamos a que $\tilde{\Phi}_e(e) = \tilde{\Phi}_e(e)+1$. Absurdo!

\end{solution}

\question (Mat.) Probar que la funci\'on $\tau : \mathrm{N} \rightarrow \mathrm{N}$ que asigna a cada n\'umero natural positivo $n$ el n\'umero de divisores positivos $n$ y $\tau(0)=0$, es una funci\'on recursiva primitiva. 

\begin{solution}
% http://www.cubawiki.com.ar/index.php?title=Final_del_21/10/14_(L%C3%B3gica_y_Computabilidad) 

Una función es primitiva recursiva si se obtiene a través de las funciones iniciales por composición y/o recursión en finitos pasos.

Sea $f:\mathbb{N}^2 \to \mathbb{N}$ tal que $f(a,b)$ devuelve la cantidad de divisores positivos desde $0$ hasta $b$. Con $a$ y $b$ naturales.

Queremos que $\tau(n) = f(n,n) \forall n \in \mathbb{N}$ definiendo a $f$ de la siguiente forma:

\begin{eqnarray*}
f(n,0)&=&n(n) \\
f(n,m+1)&=&g(n,m,f(n,m))
\end{eqnarray*}

Con 

\begin{equation*}
g(a,b,c) = \left\{ \begin{matrix} 
  suc(u^3_3(a,b,c)) & \text{si } P \\ 
  u^3_3(a,b,c) & \text{si } \neg P 
\end{matrix} \right.
\end{equation*}

Donde 

\begin{equation*}
 P = u^3_1(a,b,c) \mid suc(u^3_2(a,b,c))
\end{equation*}

Veamos que: 

\begin{enumerate}[A)]
 \setlength\itemsep{0em}
  \item $f$ cumple con el esquema de recursión primitivo.
  \item $n(n)$ es la función inicial nula aplicada a $n$.
  \item El predicado $P$ usa la función proyección y la función ``divide a'' ($\mid$) que es primitiva recursiva.
  \item La función $g$ es una división de casos disjuntos y usa las funciones iniciales de proyección y sucesor.
\end{enumerate}

Por todo lo anterior, la función $f(n,m)$ es primitiva recursiva (con $n$ y $m$ naturales).

Falta ver que $\tau (n) = f(n,n)$. Probamos por inducción en el segundo parámetro.

\begin{itemize}
 \item Caso base: 
 
 $\tau(0) = 0$
 
 $f(0,0) = n(0) = 0$ 
 
 \item Paso inductivo: 
 
 Se cumple la hipótesis inductiva $f(n,m)$ devuelve los divisiores de $n$ desde $0$ hasta $m$. Ahora queremos ver para $m+1$: $f(n,m+1) = g(n,m,f(n,m))$. Abrimos los dos casos seg\'un el dominio de $g$: 
 
  \begin{enumerate}[1.]
   \item Caso $n \mid (m+1)$: $g(n,m,f(n,m)) = f(n,m)+1$. 
   
    Entonces por H.I. al dividir $m+1$ incremento en 1 a lo ya calculado en el paso recursivo anterior y éste calculo correctamente hasta $m$. Queda $f(n,m+1)=f(n,m)+1$.
    
   \item Caso $\neg (n \mid (m+1))$: $g(n,m,f(n,m)) = f(n,m)$. 
   
    Entonces por H.I. al no dividir $m+1$ no sumo nada a lo ya calculado en el paso recursivo anterior y éste calcula correctamente hasta $m$. Queda $f(n,m+1)=f(n,m)$.
  \end{enumerate}

 Por lo tanto, $\tau(n) = f(n,n) \forall n \in \mathbb{N}$.
 
\end{itemize}

\end{solution} 

\question (Mat.) Dar un ejemplo de una funci\'on no computable de una variable tal que la imagen tenga tres elementos. 

\begin{solution}
Sea $f$: 

$$
f(x) = \left\{
\begin{array}{cl}
Halt(x,x) & \mbox{si } x \neq 0 \\
2 & \mbox{si } x = 0
\end{array}\right.
$$

La $Img(f) = \{0,1,2\}$. Tiene exactamente 3 elementos.

Supongamos que $f$ es computable, entonces $\exists P$ programa que computa $f$. 

Si $f(x) = 2$, entonces $x = 0$. 

Si $f(x) = 0 (Halt(x,x))$ o $f(x) = 1 (\neg Halt(x,x))$, si y s\'olo si $\varphi_P(x)\downarrow$.

Sea $e = \#P$. Tomemos el caso particular: $x = e$. $H(e,e)$ determina si el programa $e$ con entrada $e$ termina o no. 

Como vimos en las te\'oricas, $f$ est\'a resolviendo {\it halting problem}. Absurdo! pues {\it Halt} no es computable. Vino de suponer que $f(x)$ es computable. 

Entonces $f(x)$ no es computable. 

\end{solution}

\question Probar que TOT no es c.e. ni co-c.e.

\begin{solution}

 {\it Def. } TOT$=\{ e : \Phi_e(x)$ es total$\}$, y $\Phi_e$ es total sii $(\forall x)\Phi_e(x)\downarrow$.
 
 {\it Def. } Un conjunto $A$ es computablemente enumerable si y s\'olo si existe una funci\'on $g$ parcialmente computable, $g: \mathbb{N} \rightarrow \mathbb{N}$ tal que $A = \{x : g(x)\downarrow \} = \text{Dom}(g)$.

 {\it Def. } Un conjunto $A$ es co-c.e. sii $\overline{A}$ es c.e.
 
 {\it Demostraci\'on: } TOT no c.e.
 
 Supongamos que es c.e. $\Rightarrow \exists g$ parcial computable tal que TOT=Dom($g$).
 
 Entonces existe $e$, tal que $\Phi_e(x)=\Phi_{g(x)}(x)+1$. Notar que esto se define para todo $x$. 
 
 Como $\Phi_e$ es total, $e\in$TOT. Por lo tanto existe $u$ tal que $g(u)=e$. 
 
 Luego, $\Phi_{g(u)}(x)=\Phi_{g(u)}(x)+1$. Absurdo si $x=u$, y esto vino de suponer que TOT es c.e.
 
 {\it Demostraci\'on: } TOT no co-c.e (equivale a probar $\overline{\text{TOT}}$ no c.e).
 
 Supongamos que $\overline{\text{TOT}}$ c.e., existe entonces $d$ tal que $\overline{\text{TOT}}=$Dom($\Phi_d$). 
 
 Definimos el siguiente programa $P$: 
 
 \vspace{0.5cm}
  \begin{tabular}{rl}
    $[C]$ & IF STP$^{(1)}(X,d,T)=1$ GOTO $E$ \\
	  & $T\leftarrow T+1$\\
	  & GOTO $C$
  \end{tabular}
 \vspace{0.5cm}
 
 Tenemos
 \begin{equation*}
    \Psi_P(d,x) = g(d,x) = \left\{ \begin{matrix} 
      \uparrow & \text{si } \Phi_d \text{ es total} \\ 
      0 & \text{si no}
    \end{matrix} \right.
  \end{equation*}
  
 Por el Teorema de la Recursi\'on: sea $e$ tal que $\Phi_e(x)=g(e,x)$. 
 
 \begin{itemize}
  \item $\Phi_e$ es total $\Rightarrow\forall x: g(e,x)\uparrow\Rightarrow\forall x:\Phi_e(x)\uparrow\Rightarrow \Phi_e$ no es total.
  \item $\Phi_e$ no es total $\Rightarrow\forall x: g(e,x)=0 \Rightarrow\forall x: \Phi_e(x)=0\Rightarrow \Phi_e$ es total. 
 \end{itemize}

\end{solution}

\question Probar que todo conjunto r.e. infinito contiene un conjunto recursivo infinito. 

\begin{solution}
 
 {\it Def. } Un conjunto $A$ es recursivamente enumerable si y s\'olo si existe una funci\'on $f$ computable, $f: \mathbb{N} \rightarrow \mathbb{N}$ tal que $A = \{x : f(x)\downarrow \} = \text{Dom}(f)$.
 
 {\it Demostraci\'on: } Primero probamos que si $B = \{ f(n) : n \in \mathbb{N} \}$, con $f$ una funci\'on estrictamente creciente, parcialmente computable y total, entonces $B$ es recursivo. Para verlo, basta ver el siguiente programa: 

 \vspace{0.5cm}
  \begin{tabular}{rl}
    $[A]$ & IF $f(Z)=X$ GOTO $B$ \\
	  & IF $f(Z)>X$ GOTO $E$ \\ 
	  & $Z\leftarrow Z+1$\\
	  & GOTO $A$\\
    $[B]$ & $Y\leftarrow 1$ \\
	  & GOTO $E$
  \end{tabular}
 \vspace{0.5cm}
 
 computa $\Psi_B$. Este programa termina siempre, as\'i que $B$ es recursivo. 
 
 Ahora, sea $A$ r.e. e infinito. Tenemos una funci\'on p.r. $g$ que nos va dando los elementos de $A$. 
 
 Definimos $f(0) = g(0)$ y suponiendo definida $f(k)$ para todo $k<n$, definimos $f(n) = g(\min_z(g(z)>f(n-1)))$.
 
 Esta minimizaci\'on es propia, pues $A$ es infinito. Entonces, siempre podemos encontrar un elemento $A$ mayor que todos los que tengamos. Esto nos da una $f$ parcialmente computable, total, y estrictamente creciente. Si tomamos $B=\{f(n) : n \in \mathbb{N}\}$, tenemos que $B$ es recursivo y que $B\subseteq A$. 
\end{solution}

\question Probar que si $A$ es computable entonces es c.e. ?`Vale la vuelta? Justificar.

\begin{solution}
 Sea $P_A$ un programa para [la funci\'on caracter\'istica de] $A$. 
 
 Consideremos el siguiente programa $P$: 
 
 \vspace{0.5cm}
  \begin{tabular}{rl}
    $[C]$ & IF $P_A(X)=0$ GOTO $C$ \\
  \end{tabular}
 \vspace{0.5cm}
 
 Tenemos que: 
 
  \begin{equation*}
    \Psi_P(x) = \left\{ \begin{matrix} 
      0 & \text{si } x\in A \\ 
      \uparrow & \text{si no}
    \end{matrix} \right.
  \end{equation*}

 Y por lo tanto $A = \{x : \Psi_P(X)\downarrow\}$. 
 
 La vuelta no es cierta. Lo demostramos con un contraejemplo. 
 
 Definimos $W_n=\{ x : \Phi_n(x)\downarrow\} = $ el dominio del $n$-\'esimo programa, y a $K=\{n : n\in W_n\}$. Observemos que $n\in W_n$ sii $\Phi_n(n)\downarrow$ sii HALT($n,n$). $K$ es (a) c.e. pero (b) no computable. 
 
 \begin{enumerate}[(a)]
 \item $K$ es c.e. porque $K=\{n : \Phi_n(n)\downarrow\}$
 \item Supongamos que $K$ fuera computable. Entonces $\overline{K}$ tambi\'en lo ser\'ia. Luego existe un $e$ tal que $\overline{K}=W_e$. Por lo tanto
 \begin{equation*}
  e \in K \Leftrightarrow e \in W_e \Leftrightarrow e\in \overline{K}
 \end{equation*}

 \end{enumerate}

\end{solution}

\question Probar que un conjunto $A\subseteq \mathbb{N}$ es computable sii $A$ y $\overline{A}$ son c.e.

\begin{solution}
\begin{itemize}
 \item[$\Rightarrow$)] Si $A$ es computable entonces $\overline{A}$ es computable. 
 
 \item[$\Leftarrow$)] Supongamos que $A$ y $\overline{A}$ son c.e.
 
 \begin{equation*}
  A = \{x : \Phi_p(x)\downarrow \} \quad \overline{A} = \{x : \Phi_q(x)\downarrow \}
 \end{equation*}
 
 Consideremos el programa $P$: 
 
  \vspace{0.5cm}
  \begin{tabular}{rl}
    $[C]$ & IF STP$^{(1)}(X,p,T) = 1$ GOTO $F$ \\
	  & IF STP$^{(1)}(X,q,T) = 1$ GOTO $E$ \\
	  & $T\leftarrow T+1$ \\
	  & GOTO $C$ \\
    $[F]$ & $Y\leftarrow 1$ \\
  \end{tabular}
  \vspace{0.5cm}

  Para cada $x$, vale que $x\in A$ o bien $x\in\overline{A}$. Entonces $\Psi_P$ computa $A$. 
\end{itemize}

\end{solution}

\end{questions}